After the unprecedented technological advancements of computing over the last few decades, the goal of developing artificial intelligence (AI) systems has gained prominence. 
AI is a term that refers to intelligence demonstrated through artificial means, such as man-made machines, as opposed to the inherent intelligence found in humans.
This is Machine Learning's objective (ML).
Forging an algorithmic structure from data, in particular, to automatically enable deep insights into unknown pattern recognition and develop high-performance deterministic models.
Such systems omit the use of data-specific programming code, which has historically been the default approach when confronted with such issues.
While data-specific code may be appropriate for certain simplified models, it will fail to solve more complex problems due to its lack of versatility, i.e., it will be unable to solve a given problem without the developer applying a proven algorithm/code directly to it. 
\textcite{artifical} states ``AI  is relevant  to  any intellectual task; it is truly a universal field'', so naturally AI is used in a plethora of fields such as playing chess, proving mathematical theorems, writing poetry,  autonomous driving  and  diagnosing  diseases. A common approach for ML is the use of Artificial Neural Networks (ANNs). ANNs are a strong class of models that are inspired by biological neural computation that are frequently used for nonlinear regression and classification tasks \parencite{DREISEITL2002352}.

Although the development  of a machine with the calibre of faux human intelligence can seem far-fetched at the moment, as they can struggle to even artificially handle simple cognitive tasks designed for nursery children \parencite{deepmindchild}, advancement and implementation are considerable. As recent as 2016, the Google DeepMind team has developed a neural network-based bot called AlphaGo\footnote{AlphaGo is now the precursor to AlphaZero, a more generalised bot that can play a wider range of board games.}. Currently, AlphaGo can outperform the best human players in the historic board game Go, exhibiting to the world the potential of artificial intelligence against us humans. In a similar vein, a more sophisticated bot at Google DeepMind, AlphaStar, has been built using a deep neural network that is trained ``directly from raw game data by supervised learning and reinforcement learning'' \parencite{alphastar}; which stands above the best human players in the real-time strategy game StarCraft II. A more complicated and unbounded\footnote{Since the bot is freer, for instance, players are not restricted to fixed paths for certain units.} game than Go because AlphaStar must deal in analogy with real-world scenarios with imperfect information. An impressive representation of how a single team's construction of an AI in just two years has evolved\footnote{Readers may be interested by OpenAI's GPT-3 deep learning autoregressive language model, another excellent example of AI with great future potential.}. 